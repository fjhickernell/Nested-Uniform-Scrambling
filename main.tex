\documentclass{amsart}
\usepackage{mathtools,upref,siunitx,upquote,fancyvrb,color}
\usepackage[hyphens]{url}
\usepackage[utf8]{inputenc}
\input{FJHDef.tex}


\usepackage{algpseudocode}
\usepackage{algorithm, algorithmicx}
\algnewcommand\algorithmicparam{\textbf{Parameters:}}
\algnewcommand\PARAM{\item[\algorithmicparam]}
\algnewcommand\algorithmicinput{\textbf{Input:}}
\algnewcommand\INPUT{\item[\algorithmicinput]}
\algnewcommand\RETURN{\State \textbf{Return }}



\begin{document}
\title{Nested Uniform Scrambling}
\author{Fred J. Hickernell}
\begin{abstract}This project is where all of the files go that are needed elsewhere
\end{abstract}

\maketitle

Suppose that $a_0, a_1, \ldots$ is a \textbf{one-dimensional} digital sequence in base $2$, and we went to implement nested uniform scrambling (NUS) \cite{Owe95}.  Let $\va_i = (a_{i1}, a_{i2}, \ldots, a_{iM})$ denote the  vector of binary digits of $a_i$.  The NUS version of the sequence, $x_0, x_1, \ldots$ with digits $\vx_i = (x_{i1}, x_{i2}, \ldots, \vx_{iM})$ is defined as
\begin{equation*}
    x_{ik} = \pi_{\bullet,\va_{i,1:k-1}}(a_{ik}),
\end{equation*}
where $\va_{i,1:k-1} = (a_{i1}, \ldots, a_{ik-1})$, and all the $\pi_{\bullet,\bullet}$ are IID uniform over $\{0,1\}$.

Although NUS can be applied to any $(t,m,s)$-sequence, we will apply it to digital sequences.  This means that 
\begin{equation*}
    \va_i = i_0 \va_1 \oplus i_1 \va_2 \oplus i_2 \va_4 \oplus \cdots \oplus i_m \va_{2^m} \oplus \cdots, \quad \text{where } i = i_0  + 2 i_1 + 4 i_2 + \cdots.
\end{equation*}

Suppose that we just have $M = 5$ bit precision
\begin{align*}
    \va_0 & = (0, 0,0,0,0) \\
    \va_1 & = (1, 1, 0, 0, 1) \\
    \va_2 & = (1, 0, 1, 0, 1) \\
    \va_3 & = \va_1 \oplus \va_2 = (0, 1, 1, 0, 0)
\end{align*}
And then
\begin{align*}
    \vx_0 & = ({\color{blue} 0, 1, 1, 1, 0}) & \text{{\color{blue}purely random digits}}\\
    \vx_1 & = (1 - 0 = 1, {\color{blue}0, 0, 1, 1}) & \text{because } a_{1,1} &= 1 - a_{0,1}\\
    \vx_2 & = (1, 1 - 0 = 1, {\color{blue}1, 0, 1}) & \text{because }\va_{2,1:1} = \va_{1,1:1}, \  a_{22} &= 1 - a_{1,2} \\
    \vx_3 & = (0, 1-1 = 0, {\color{blue}1, 0, 0}) & \text{because }\va_{3,1:1} = \va_{0,1:1}, \  a_{32} &= 1 - a_{0,2} \\
\end{align*}


If $i \in \{0, \ldots, 2^{m}-1\}$ and $j \in \{2^m, \ldots, 2^{m+1} - 1\}$ satisfy \begin{equation} \label{eq:ijeq}
    \va_{j,1:m-1} = \va_{i,1:m-1} \oplus \va_{2^m,1:m-1},
\end{equation}
then we know that
\begin{equation*}
    \vx_{j} = (x_{i1}, x_{i2}, \ldots, x_{i,m-1}, 1-x_{i,m}, x_{j,m+1}, x_{j,m+2},\ldots),
\end{equation*}
where the $x_{j,k}, \ldots$ are $\IIDsim \cu\{0,1\}$.  For example, 
\begin{align*}
    m & = 0 & i &= 0 \in \{0\}, & j &= 1 \in \{1\}, 
    & \vx_1 &= (1 - x_{01} = 1, {\color{blue}0, 0, 1, 1}) \\
    m & = 1 & i &= 1 \in \{0,1\}, & j &= 2 \in \{2,3\}, 
    & \vx_2 &= (x_{11} = 1, 1 - x_{12} = 1, {\color{blue}1, 0, 1}) \\
    &  & i &= 0 \in \{0,1\}, & j &= 3 \in \{2,3\}, 
    & \vx_3 &= (x_{01} = 0, 1 - x_{02} = 0, {\color{blue}1, 0, 1}) \\
\end{align*}

How do we determine $i$ and $j$ satisfying \eqref{eq:ijeq}?
\begin{align*}
    \va_{i} = i_0 \va_1 \oplus \cdots \oplus i_{m-1} \va_m
\end{align*}
Define
\begin{equation*}
    \mA_{mn} = \begin{pmatrix}
        a_{11} & a_{21} & \cdots & a_{n1} \\
        a_{12} & a_{22} & \cdots & a_{n2} \\
        \vdots & \vdots & \ddots & \vdots \\
        a_{1m} & a_{12} & \cdots & a_{nm} 
    \end{pmatrix}, \qquad
    \vi_m = \begin{pmatrix}i_0 \\ i_1 \\ \vdots \\ i_{m-1}\end{pmatrix}
\end{equation*}
Then \eqref{eq:ijeq} can be written as 
\begin{equation*}
    \mA_{m-1,m-1} \vi_{m-1} = \mA_{m-1,m} \vj_{m}
\end{equation*}


This implies that 
\begin{equation*}
    \va_{i+2^m} = \va_i \oplus \va_{2^m}  \qquad \text{for } i \in \{0, \ldots, 2^{m}-1\}, \quad m = 0, 1, \ldots.
\end{equation*}


\bibliographystyle{amsplain}
\bibliography{FJH23,FJHown23}



\end{document}
